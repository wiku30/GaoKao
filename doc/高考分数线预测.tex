%% LyX 2.1.4 created this file.  For more info, see http://www.lyx.org/.
%% Do not edit unless you really know what you are doing.
\documentclass{article}
\usepackage{amsmath}
\usepackage{fontspec}
\setmainfont[Mapping=tex-text]{宋体}
\usepackage{geometry}
\geometry{verbose,tmargin=2.5cm,bmargin=2.5cm,lmargin=2.5cm,rmargin=2.5cm}
\usepackage{xunicode}
\begin{document}

\title{高考志愿填报建议系统}


\author{清华大学交叉信息院 赵梓硕}

\maketitle

\section{简介}


\subsection{作者信息}

赵梓硕,本科生,就读于清华大学交叉信息研究院计科50班。E-mail: 741488923@qq.com


\subsection{项目简介}

每年高考后,填报志愿都是考生们的一大难题。由于每年高考的难易度和区分度都有所不同,每年各大高校及专业的录取分数线也有较大波动。因此,在估计高校的录取线时,一般人往往以历年录取线与批次线的线差作为重要依据。本项目正是以此思想为基础,试图建立一个更加合理的数学模型,对高校分数线给出一个相对更加准确的估计,并以此为依据,根据考生所在地区、高考分数和其偏好,为其推荐合适的高校与专业。

此项目为作者大一暑假实习内容。由于部分数据(如各地区历年考生分布)需要手工获取,以及实习时间、运算资源有限等原因,此项目暂只对北京、湖北、黑龙江、辽宁四个地区理科一本分数段的考生适用。


\section{建模部分}


\subsection{原理}

根据我国高招的规则,无论是顺序志愿或是平行志愿,高校的都是在符合条件的考生中按分数从高到低排序录取的,录取的最后一名考生分数即是该院校(专业)的分数线。因此,当院校与专业的热门程度不变的前提下,相比录取线本身,其对应的位次相对更为稳定。

考试的最终目的是通过分数衡量考生水平,并选拔水平符合要求的考生。位次可以体现考生在总体中的相对水平,然而由于分布的非线性性,位次与水平成非线性关系。同时,正如熟知的“从95分到100分,远比从70分到75分要难”,即便排除了考试题的不确定性,考试的预期分数与实际水平也呈现非线性关系。此数学模型要得出的则是一个相对稳定的标尺,衡量考生水平与各院校的要求,从而给出合理的建议。


\subsection{数学模型}

每年高考中存在很多不确定的因素,在进行建模的过程中,作者进行了以下假设:

(1)考生实际水平$L$呈正态分布,且每年保持稳定。

(2)考试成绩$S$准确反映了考生的水平,即考分是水平的一个保序变换。

(3)考生总体意愿每年相同,各院校专业录取“水平线”在期望值附近作无规律波动。

参照四六级考试的“标准分”机制,以一分一段表为依据,若考生$i$成绩$S_{i}$对应的位次为$k_{i}$,而当年该省考生总人数为$n_{i}$,则定义相对位次

\[
\alpha_{i}=\dfrac{n_{i}-k_{i}}{n_{i}}.
\]


在忽略并列影响的前提下,对于任意考生$j$,有

\[
\Pr[S_{i}>S_{j}]=\alpha.
\]


根据假设(2),有

\[
\Pr[L_{i}>L_{j}]=\alpha.
\]


根据假设(1),并参考人类智力分布特征,不妨设(此步参数与最终结果无关,但本系统及此文档中均使用此度量)

\[
L\sim N(100,\,15^{2}),
\]


那么已知一分一段表与成绩$S_{i}$,即可得出对应水平$L_{i}$。


\subsection{模型简化}

在实际操作中,要由$S_{i}$得出$L_{i}$,必须要该省当年完整的一分一段表,然而各个省一分一段表的范围、格式往往不同,难以在同一个网站上批量取得,甚至许多还以图片形式展现,因而难以使用程序批量处理,需要手工获取。然而,一张一分一段表往往有数百行,再考虑数十个省的历年情况,如果一一手工输入,工作量会极为繁重。因此,在实际操作中,需要对一分一段表表现的的分布状况进行拟合。

本项目中主要考虑一本线上的考生。对于较为优秀的学生,其对高中知识应当已基本掌握,因而高考对该群体而言“不是比谁得分多,是比谁丢分少”。其能力与训练程度(这里统称为水平)的不同,导致了其面对高考灵活多变的题型时,失误的程度会有所不同,从而最终得分有所不同。

这里假设满分为$\Omega$,定义失误率

\[
\varepsilon=\dfrac{\Omega-S}{\Omega}.
\]


显然$\varepsilon$是关于$L$的减函数。然而$\varepsilon$与$L$并不成线性关系。由常识可知,当$\varepsilon$已经足够小后,进一步减小它的值将极为困难。由附件Excel表格Sheet1(数据来自北京市近三年一分一段表)可见,其斜率绝对值随$L$的增加而减小,与预期相吻合。

根据四省近3\textasciitilde{}4年的数据,$\varepsilon$与$L$的关系大致为

\[
\varepsilon^{\lambda}\approx kL+b.
\]


取$\lambda=-0.35$(如Sheet2图表可见),对北京市各年数据都有较好的拟合,但由于各地区试卷风格不同,加上同一地区各年亦有微小差异,根据具体分布来确定$\lambda$值,拟合效果会更好。具体操作只需取若干个$L$值,调整不同的$\lambda$值作线性回归,使相关系数$r$最大即可。(事实上$\lambda$值一定的偏差对最终结果影响并不大,精确至0.05即足够)

在此四省中,最优$\lambda$值在$-0.05$至$-0.45$之间不等。

经实际测试,在$L$不小于106\textasciitilde{}110,不大于150左右时,此拟合较为精确。因此该模型对排名前30\%左右的考生有效($L>150$的考生往往对应清华、北大或至少复旦、交大的档次,并不需要太多建议),可用于一本院校分数线预测。


\section{具体操作}

首先,需要得出各省各年高考分数转换对应表。只须在Sheet3中的“总人数”和“满分”中填入该次高考情况,表格中“名次”一栏就会自动显示$L=150,140,130,120,110$对应的名次,然后在“分数”栏对应位置填写该名次对应分数,并调整“参数”栏数据(精确至0.05即可)使右侧图表中的$R^{2}$最大,再将Sheet4的内容复制至univ/std/xx-yyyy文件下(xx为省代号,yyyy为年份),即录入了该省当年分数与水平值的对应表。

运行univ/data/getU.py,即可在网站http://gkcx.eol.cn/上抓取各高校在各地区的录取平均分。(网站上无最低分,由于自主招生等原因,最低分也无太大意义,因此结果会偏高,但事实上踩线录取一般会进入冷门专业,因此以平均分为基准有合理性)数据保存在在univ/data/下,格式为res-xx(xx为省代号)。

运行univ/make.sh,可以将此项目的C++程序进行编译。

运行univ/c-bin/list-generator,可生成各省份各高校平均分预测,数据保存在univ/pred/下,格式为xx-{*}{*}{*}{*}{*}(xx为省代号,{*}{*}{*}{*}{*}为省名拼音,如00-BeiJing)。

运行univ/spec-xml/getS.py,可抓取各专业xml信息并保存在spec-xml/data下,再运行univ/spec-xml/proc.py,可处理其信息并保存在spec-xml/processed下,文件名为xx(xx为省代号)。

运行univ/tags/tags,可由univ/data/res-xx生成各高校标签信息,保存在univ/tags/data/下。

将website/univ/header.php中的root\_path改为univ目录绝对路径,即可通过website/univ/index.html运行该项目。


\section{源代码路径}

univ/data/getU.py, univ/spec-xml/getS.py, univ/spec-xml/proc.py:获得数据

univ/c-bin/list-generator.cpp:生成院校分数预测

univ/c-bin/search.cpp:查询符合条件的院校

univ/tags/tags.cpp:生成学校标签数据

univ/spec-xml/pred.cpp:查询符合条件的专业(-u为按学校查询,-s为按专业查询)
\end{document}
